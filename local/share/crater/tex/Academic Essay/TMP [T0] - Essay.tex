\documentclass[11pt]{article} % Define the document type and font size

\usepackage[spanish]{babel} % Package for the Spanish language
\usepackage[utf8]{inputenc} % Package for UTF-8 character encoding
\usepackage{geometry} % Package to configure document margins
\usepackage{listings} % Package to include source code in the document
\usepackage{xcolor} % Package to define colors
\usepackage{fancyhdr} % Package to customize headers and footers
\usepackage{amsmath} % Package for advanced mathematical symbols and environments
\usepackage{amssymb} % Package for additional mathematical symbols
\usepackage{url} % Package to format URLs
\usepackage{hyperref} % Package to create hyperlinks
\usepackage{multicol} % Package for multiple columns
\usepackage{caption} % Package to customize captions
\usepackage{listings} % Package to include source code in the document
\usepackage{graphicx} % Package to include images
\usepackage{xcolor} % Package to define colors

% Font configuration
\usepackage[T1]{fontenc} % Package for font encoding
\usepackage{stix2} % Package for STIX2 fonts
\usepackage{FiraMono} % Package for Fira Mono font
\let\oldtexttt\texttt
\renewcommand{\texttt}[1]{{\small\oldtexttt{#1}}} % Redefine \texttt to use a smaller font size
\captionsetup{hypcap=false}

\geometry{ % Configuration of document margins
	a4paper,
	left=25mm,
	right=25mm,
	top=25mm,
	bottom=25mm
}

\setlength{\parskip}{1em} % Space between paragraphs
\setlength{\parindent}{0pt} % Without indentation for paragraphs

% Definition of colors for source code
\definecolor{COLOR_COMMENTS}{HTML}{7C6F64}
\definecolor{COLOR_KEYWORDS}{HTML}{CC241D}
\definecolor{COLOR_EMPH}{HTML}{076678}
\definecolor{COLOR_STRINGS}{HTML}{98971A}
\definecolor{COLOR_CONSOLE}{HTML}{EEEEEE}

\hypersetup{
	colorlinks=true, % Enable colored links
	linkcolor=COLOR_EMPH, % Color for internal links (e.g., TOC)
	urlcolor=COLOR_EMPH, % Color for external links (e.g., references)
	citecolor=COLOR_EMPH % Color for citations
}

\title{
	\vspace*{-2.5cm}
	\begin{minipage}{0.15\textwidth} \includegraphics[width=\textwidth]{TMP [T0] - Recursos/logo_ucv.png} \end{minipage}
	\hfill
	\begin{minipage}{0.15\textwidth} \includegraphics[width=\textwidth]{TMP [T0] - Recursos/logo_ciens.png} \end{minipage} \\ [1.5cm]
	\textbf{Título del Ensayo Académico} \\
	{\large{Análisis de Métricas y Símbolos en el entorno \LaTeX}}
}
\author{\textit{Nombre Apellido, Nombre Apellido}} % Reduce espacio antes del autor
\date{\vspace{-1.5em}\today} % Reduce espacio antes de la fecha

\begin{document}

\maketitle

\begin{abstract}
¿¡"Resumen Experimental"!? El abstract contiene: @símbolos, \#etiquetas, \$monedas y  también los \%porcentajes. —Este texto— busca demostrar el uso de «comillas latinas» y ``comillas inglesas'', además de (paréntesis), [corchetes] y {llaves}. La integración de \& (ampersand) y * (asteriscos) es crucial para la validación del 100\% de los datos analizados/observados.
\end{abstract}

\section{Desarrollo}

\begin{multicols}{2}
	¿Lorem ipsum "dolor" sit amet, 'consectetuer' adipiscing elit? «Aenean» commodo ligula \& eget dolor \#1. Aenean massa; Cum sociis natoque penatibus et magnis (dis) parturient montes, nascetur ridiculus mus! Donec quam felis, 50\% de ultricies nec, pellentesque eu, pretium quis, sem.

	\begin{equation} \label{eq:einstein_lorem}
		R_{\mu\nu} - \frac{1}{2}R g_{\mu\nu} + \Lambda g_{\mu\nu} = \frac{8\pi G}{c^4} T_{\mu\nu}
	\end{equation}

	Aenean imperdiet. Etiam ultricies nisi vel augue. Curabitur ullamcorper ultricies nisi. Nam eget dui. Etiam rhoncus. Maecenas tempus, tellus eget condimentum rhoncus, sem quam semper libero, sit amet. Nulla consequat massa quis enim. Donec pede justo, fringilla vel, aliquet nec, vulputate eget, arcu.

	La Ecuación \ref{eq:einstein_lorem} describe el ``Lore Ipsum'' del espacio-tiempo: ¡In enim justo!, rhoncus ut, imperdiet a, “venenatis” vitae, justo. Nullam dictum felis eu pede mollis pretium. Integer tincidunt. In enim justo, rhoncus ut, imperdiet a, venenatis vitae, justo. Nullam dictum felis eu pede mollis pretium. ¿Phasellus viverra nulla ut metus varius laoreet? Quisque rutrum: ``Aenean imperdiet''. Etiam ultricies nisi vel augue (ca. 10--20 unidades). Curabitur ullamcorper ultricies nisi; Nam eget dui.
\end{multicols}

\begin{center}
	\begin{equation}
		\mathcal{Z} = \int \mathcal{D}\phi \exp \left\{ \frac{i}{\hbar} \int d^4x \left[ \frac{1}{2} (\partial_\mu \phi)^2 - \frac{1}{2} m^2 \phi^2 - \frac{\lambda}{4!} \phi^4 \right] \right\}
	\end{equation}
	\captionof{figure}{Fórmula para calcular el significado de la Vida}
	\vspace{1cm}
\end{center}

\newpage

\section{Referencias Bibliográficas}
\vspace{-4em}
\renewcommand{\refname}{} % Remove the default bibliography title
\renewcommand{\bibname}{} % For book/report classes
\begin{thebibliography}{99}

\bibitem{wiki_ansi}
Wikipedia. (2025, 15 de septiembre). \textit{ANSI C}. \href{https://es.wikipedia.org/wiki/ANSI\_C}{https://es.wikipedia.org/wiki/ANSI\_C}

\end{thebibliography}

\end{document}
