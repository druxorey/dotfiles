\documentclass[12pt]{article} % Define the document type and font size

\usepackage[spanish]{babel} % Package for the Spanish language
\usepackage[utf8]{inputenc} % Package for UTF-8 character encoding
\usepackage{geometry} % Package to configure document margins
% \usepackage{fancyhdr} % Package to customize headers and footers
% \usepackage{multicol} % Package for multiple columns
\usepackage{amsmath} % Package for advanced mathematical symbols and environments
\usepackage{amssymb} % Package for additional mathematical symbols
\usepackage{graphicx} % Package to include images
\usepackage{xcolor} % Package to define colors
\usepackage{listings} % Package to include source code in the document
\usepackage{url} % Package to format URLs
\usepackage{hyperref} % Package to create hyperlinks
% \usepackage{float} % Package for improved interface for floating objects
\usepackage{caption} % Package to customize captions
% \usepackage{booktabs} % Package for professional quality tables
% \usepackage{array} % Package for advanced table formatting

% Font configuration
\usepackage[T1]{fontenc} % Package for font encoding
\usepackage{stix2} % Package for STIX2 fonts
\usepackage{FiraMono} % Package for Fira Mono font
\let\oldtexttt\texttt
\renewcommand{\texttt}[1]{{\small\oldtexttt{#1}}} % Redefine \texttt to use a smaller font size
\captionsetup{hypcap=false}

\geometry{ % Configuration of document margins
	a4paper,
	left=25mm,
	right=25mm,
	top=25mm,
	bottom=25mm
}

\setlength{\parskip}{1em} % Space between paragraphs
\setlength{\parindent}{0pt} % Without indentation for paragraphs

% Definition of colors for source code
\definecolor{COLOR_COMMENTS}{HTML}{7C6F64}
\definecolor{COLOR_KEYWORDS}{HTML}{CC241D}
\definecolor{COLOR_EMPH}{HTML}{076678}
\definecolor{COLOR_STRINGS}{HTML}{98971A}

\hypersetup{
	colorlinks=true, % Enable colored links
	linkcolor=COLOR_EMPH, % Color for internal links (e.g., TOC)
	urlcolor=COLOR_EMPH, % Color for external links (e.g., references)
	citecolor=COLOR_EMPH % Color for citations
}

% Configuration of the listings environment to display source code
\lstdefinestyle{base}{
	frame=shadowbox,
	aboveskip=10mm,
	xleftmargin=5mm,
	basicstyle={\footnotesize\ttfamily},
	numberstyle=\tiny\color{COLOR_COMMENTS},
	commentstyle=\color{COLOR_COMMENTS},
	keywordstyle=\color{COLOR_KEYWORDS},
	emphstyle=\color{COLOR_EMPH},
	stringstyle=\color{COLOR_STRINGS},
	breaklines=true,
	breakatwhitespace=true,
	showstringspaces=false,
	columns=flexible,
	numbers=left,
	keepspaces=true,
	tabsize=4,
	captionpos=b
}

\lstdefinestyle{c}{
	style=base,
	language=c,
	morekeywords={NULL},
	emph={int,long,short,float,double,char,bool,void,size_t,ssize_t,uint8_t,
		uint16_t,uint32_t,uint64_t,int8_t,int16_t,int32_t,int64_t,pthread_t,
		pthread_attr_t,pthread_mutex_t,pthread_mutexattr_t,pthread_cond_t,
		pthread_condattr_t,sem_t,sig_atomic_t,sigset_t,atomic_flag,pid_t,
		off_t,time_t},
    literate=
		{=}{{\textcolor{COLOR_KEYWORDS}{=}}}1
		{*}{{{\textcolor{COLOR_KEYWORDS}{*}}}}1
		{&}{{\textcolor{COLOR_KEYWORDS}{\&}}}1
		{+}{{\textcolor{COLOR_KEYWORDS}{+}}}1
		{-}{{\textcolor{COLOR_KEYWORDS}{-}}}1
		{<}{{\textcolor{COLOR_KEYWORDS}{<}}}1
		{>}{{\textcolor{COLOR_KEYWORDS}{>}}}1
		{!}{{\textcolor{COLOR_KEYWORDS}{!}}}1
		{.}{{\textcolor{COLOR_KEYWORDS}{.}}}1
		{::}{{\textcolor{COLOR_KEYWORDS}{::}}}2
		{->}{{\textcolor{COLOR_KEYWORDS}{->}}}2
}

\lstdefinestyle{cpp}{
	style=base,
	language=c++,
	morekeywords={std},
	emph={int,long,short,float,double,char,bool,void,size_t,ssize_t,uint8_t,
		uint16_t,uint32_t,uint64_t,int8_t,int16_t,int32_t,int64_t,string,
		vector,i32,i64,f32,f64},
    literate=
		{=}{{\textcolor{COLOR_KEYWORDS}{=}}}1
		{*}{{{\textcolor{COLOR_KEYWORDS}{*}}}}1
		{&}{{\textcolor{COLOR_KEYWORDS}{\&}}}1
		{+}{{\textcolor{COLOR_KEYWORDS}{+}}}1
		{-}{{\textcolor{COLOR_KEYWORDS}{-}}}1
		{<}{{\textcolor{COLOR_KEYWORDS}{<}}}1
		{>}{{\textcolor{COLOR_KEYWORDS}{>}}}1
		{!}{{\textcolor{COLOR_KEYWORDS}{!}}}1
		{.}{{\textcolor{COLOR_KEYWORDS}{.}}}1
		{::}{{\textcolor{COLOR_KEYWORDS}{::}}}2
		{->}{{\textcolor{COLOR_KEYWORDS}{->}}}2
}

\lstdefinestyle{python}{
	style=base,
	language=Python,
	morekeywords={...},
	emph={int,float,str,bool,list,tuple,dict,set,None}
}

\lstdefinestyle{bash}{
	style=base,
	language=Bash,
	morekeywords={...},
	emph={...},
	literate=
		{=}{{\textcolor{COLOR_KEYWORDS}{=}}}1
		{\$}{{\textcolor{COLOR_KEYWORDS}{\$}}}1
		{@}{{\textcolor{COLOR_KEYWORDS}{@}}}1
		{\&\&}{{\textcolor{COLOR_KEYWORDS}{\&\&}}}2
		{||}{{\textcolor{COLOR_KEYWORDS}{||}}}2
}

\lstdefinestyle{consola}{
	style=base,
	keepspaces=true,
	numbers=none,
	xleftmargin=10mm,
	xrightmargin=10mm,
}

\title{Título del Informe} % Document title
\author{Nombre} % Document author
\date{\today} % Document date

\begin{document}

\begin{titlepage}
	\centering
	\begin{center}
		\includegraphics[width=3cm]{TMP [T0] - Recursos/logo_ucv.png}
	\end{center}
	{\vspace{-1.5em} \large {UNIVERSIDAD CENTRAL DE VENEZUELA}\par}
	{\vspace{-1em} \large {FACULTAD DE CIENCIAS}\par}
	{\vspace{-1em} \large {ESCUELA DE COMPUTACIÓN}\par}
	{\vspace{-1em} \large {MATERIA}\par}
	\vspace{6cm}
	{\LARGE \textbf{TÍTULO DEL INFORME}\par}
	{\Large INFORME DE LABORATORIO\par}
	\vfill
	{\large Nombre Apellido, Nombre Apellido\par}
	{\large \today\par}
\end{titlepage}

\setcounter{tocdepth}{2} % Depth of the table of contents
\tableofcontents % Table of contents

\newpage

\section{Objetivos}

\subsection{Objetivo General}

\begin{itemize}
    \item Comprender los procesos y su comunicación en sistemas operativos.
\end{itemize}

\subsection{Objetivos Específicos}

\begin{itemize}
    \item Entender el concepto de proceso y su estructura
    \item Entender las diferencias entre las llamadas al sistema \texttt{fork()} y \texttt{vfork()}
    \item Conocer las herramientas disponibles en los sistemas operativos para realizar comunicación entre procesos
    \item Adquirir destrezas en el desarrollo de soluciones, basadas en la bifurcación de instrucciones y comunicación entre procesos
\end{itemize}

\newpage

\section{Marco Teórico}

¿Quisquam «est qui ``dolorem 'ipsum' quia'' dolor» sit amet? Elit @consectetur [adipiscing] vélit, sed quia non-numquam (eius) modi 15\% tempora incidunt ut labore \& dolore magnam \{aliquam\} quaerat voluptatem. Ut enim ad mínima veniam, quis nostrum \$10.99 exercitationem ullam corporis suscipit laboriosam, nisi ut aliquid ex ea commodi \^{}consequatur? Quis autem vel eum iure reprehenderit qui in ea voluptate velit esse quam nihil molestiae \_consequatur, vel illum qui dolorem eum fugiat $\aleph_0$ quo voluptas nulla \#pariatur! Nam libero tempore, cum soluta nobis est eligendi <optio> cumque nihil $\nabla$ impedit quo minus id quod maxime placeat facere possimus, omnis voluptas [assumenda] est, omnis $\oiiint$ dolor repellendus. Temporibus autem quibusdam et aut officiis debitis aut rerum necessitatibus \~{s}aepe\~{} eveniet ut et voluptates repudiandae sint et molestiae non recusandae.

\subsection{Maecenas Tempus Aellus Eget}

Nam libero tempore, cum soluta nobis est eligendi <optio> cumque nihil impedit quo minus id quod maxime placeat facere possimus, omnis voluptas [assumenda] est, omnis dolor repellendus. Itaque earum rerum hic tenetúr a sapiente delectus, ut aut reiciendis voluptatibus maiores alias consequatur aut perferendis doloribus asperiores repellat. ¿Aenean massa cum sociis natoque \{nascetur\} ridiculus mus? Donec quam felis, 50\% de [ultricies] nec, pellentesque eu, pretium quis, sem. Eu fugiat du x > y \&\& z < w; ¿verdol? Ea vaxianza $\sigma^2$ enim justo, rhoncus ut, imperdiet a [0, 100] scelerisque ut, mollis sed f(x) $\rightarrow$ \{y\} vitae sapien ut libero venenatis @metadata.

\begin{itemize} \item \textbf{«Aenean» commodo ligula:} Eget dolor \#1. Aenean massa; Cum sociis natoque penatibus et magnis (dis) parturient montes, nascetur $\wp$ ridiculus mus! Donec quam felis, 50\% de [ultricies] nec, pellentesque eu, pretium quis, sem. Nulla consequat $\complement$ massa quis enim. Donec pede justo, fringilla $\exists!$ vel, aliquet nec, vulputate $\nexists$ eget, arcu. In enim justo, rhoncus ut, imperdiet a, venenatis vitae, justo.

\item \textbf{Nullam dictum felis:} Eu pede mollis pretium. Integer tincidunt. Cras dapibus. Vivamus elementum semper nisi. Aenean vulputate eleifend tellus. Aenean leo $\because$ ligula, porttitor eu, consequat vitae, eleifend $\therefore$ ac, enim.

\item \textbf{Aliquam lorem ante:} Sapibus in, viverra quis, feugiat a, tellus. Phasellus viverra nulla ut metus varius laoreet. Quisque rutrum. Aenean imperdiet. Etiam ultricies nisi vel augue. Curabitur ullamcorper ultricies nisi. Nam eget dui.

\end{itemize}

\newpage

\section{Desarrollo}

Etiam rhoncus $\beta \approx 0.5$. Maecenas tempus, tellus eget condimentum $\alpha_0$ rhoncus, sem quam semper libero $\sum_{j=0}^{\infty} \chi_j$, sit amet adipiscing sem neque sed ipsum.

\begin{lstlisting}[style=cpp, caption={Ejemplo de Código de C++}]
#include <vector>
#include <algorithm>

int main(int argc, char *argv[]) {
	std::vector<i64> numbers = {5, 3, 8, 6, 2};
	std::sort(numbers.begin(), numbers.end());
	// Apply a lambda function to print the square of each number
	std::for_each(numbers.begin(), numbers.end(), [](i64 x) {
		std::cout << x * x << " ";
	});
}
\end{lstlisting}

¿Quisquam «est qui ``dolorem 'ipsum' quia'' dolor» sit amet? Elit @consectetur [adipiscing] vélit, sed quia non-numquam (eius) modi 15\% tempora incidunt ut labore \& dolore magnam \{aliquam\} quaerat voluptatem. Ut enim ad mínima veniam, quis nostrum \$10.99 exercitationem ullam corporis suscipit laboriosam, nisi ut aliquid ex ea commodi \^{}consequatur? Quis autem vel eum iure reprehenderit qui in ea voluptate velit esse quam nihil molestiae \_consequatur. «¡Nam quam nunc, blandit @vel, 'luctus' pulvinar!», hendrerit id, lorem \#23. Maecenas nec odio et ante tincidunt tempus [75\%]. Donec vitae sapien ut libero venenatis faucibus $\{ \mu, \sigma \}$.

\begin{lstlisting}[style=python, caption={Ejemplo de Código de Python}]
def factorial(n: int) -> int:
	# Base case: factorial of 0 or 1 is 1
	if n == 0 or n == 1:
		return 1
	return n * factorial(n - 1)

if __name__ == "__main__":
	num: int = int(input("Enter a number: "))
	print(f"The factorial of {num} is {factorial(num)}")
\end{lstlisting}

Nullam quis ante $\neq$ null. Etiam sit amet orci eget eros faucibus tincidunt $\int_a^b f(x)dx$. Duis leo $\Rightarrow$ «Ep "miño 'aoñí' comp" ia bigünña».

\begin{lstlisting}[style=consola, caption={Resultado de ejecución del programa \texttt{r1.c}}]
llamando read(3, c, 10).  valor de retorno: 10 bytes leidos
se imprimen los bytes leidos: Jim Plank
\end{lstlisting}

\newpage

\section{Conclusión}

Sed fringilla mauris sit amet nibh. Donec sodales sagittis magna. Sed consequat, leo eget bibendum sodales, augue velit cursus nunc, quis gravida magna mi a libero. Fusce vulputate eleifend sapien $\Omega_{total}$. Vestibulum purus quam, scelerisque ut, mollis sed, nonummy id, metus. Nullam accumsan lorem in dui $\pm 1.96$. Cras ultricies mi eu turpis hendrerit fringilla. Vestibulum ante ipsum primis in faucibus orci luctus et ultrices posuere cubilia Curae; In ac dui quis mi consectetuer lacinia.

\begin{figure}[h]
	\centering
	\includegraphics[width=0.75\textwidth]{TMP [T0] - Recursos/img_0001.png}
	\caption{Descripción de la imagen}
\end{figure}

\newpage

\section{Referencias Bibliográficas}
\vspace{-4em}
\renewcommand{\refname}{} % Remove the default bibliography title
\renewcommand{\bibname}{} % For book/report classes
\begin{thebibliography}{99}
\hbadness=10000
\sloppy
\urlstyle{same}

\bibitem{downey2009} Downey, A. B. (2009). \textit{The Little Book of Semaphores}. \url{http://greenteapress.com/semaphores/index.html}

\end{thebibliography}

\end{document}
