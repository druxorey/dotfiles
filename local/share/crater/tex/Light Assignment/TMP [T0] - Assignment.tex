\documentclass[12pt]{article} % Define the document type and font size

\usepackage[spanish]{babel} % Package for the Spanish language
\usepackage[utf8]{inputenc} % Package for UTF-8 character encoding
\usepackage{geometry} % Package to configure document margins
\usepackage{listings} % Package to include source code in the document
\usepackage{xcolor} % Package to define colors
\usepackage{fancyhdr} % Package to customize headers and footers
\usepackage{amsmath} % Package for advanced mathematical symbols and environments
\usepackage{amssymb} % Package for additional mathematical symbols
\usepackage{url} % Package to format URLs
\usepackage{hyperref} % Package to create hyperlinks
\usepackage{multicol} % Package for multiple columns
\usepackage{caption} % Package to customize captions
\usepackage{listings} % Package to include source code in the document
\usepackage{graphicx} % Package to include images
\usepackage{xcolor} % Package to define colors

% Font configuration
\usepackage[T1]{fontenc} % Package for font encoding
\usepackage{stix2} % Package for STIX2 fonts
\usepackage{FiraMono} % Package for Fira Mono font
\let\oldtexttt\texttt
\renewcommand{\texttt}[1]{{\small\oldtexttt{#1}}} % Redefine \texttt to use a smaller font size
\captionsetup{hypcap=false}

\geometry{ % Configuration of document margins
	a4paper,
	left=25mm,
	right=25mm,
	top=25mm,
	bottom=25mm
}

\setlength{\parskip}{1em} % Space between paragraphs
\setlength{\parindent}{0pt} % Without indentation for paragraphs

% Definition of colors for source code
\definecolor{COLOR_COMMENTS}{HTML}{7C6F64}
\definecolor{COLOR_KEYWORDS}{HTML}{CC241D}
\definecolor{COLOR_EMPH}{HTML}{076678}
\definecolor{COLOR_STRINGS}{HTML}{98971A}
\definecolor{COLOR_CONSOLE}{HTML}{EEEEEE}

\hypersetup{
	colorlinks=true, % Enable colored links
	linkcolor=COLOR_EMPH, % Color for internal links (e.g., TOC)
	urlcolor=COLOR_EMPH, % Color for external links (e.g., references)
	citecolor=COLOR_EMPH % Color for citations
}

% Configuration of the listings environment to display source code
\lstdefinestyle{base}{
	frame=shadowbox,
	aboveskip=5mm,
	xleftmargin=5mm,
	basicstyle={\footnotesize\ttfamily},
	numberstyle=\tiny\color{COLOR_COMMENTS},
	commentstyle=\color{COLOR_COMMENTS},
	keywordstyle=\color{COLOR_KEYWORDS},
	emphstyle=\color{COLOR_EMPH},
	stringstyle=\color{COLOR_STRINGS},
	breaklines=true,
	breakatwhitespace=true,
	showstringspaces=false,
	columns=flexible,
	numbers=left,
	tabsize=4,
	captionpos=b
}

\lstdefinestyle{cpp}{
	style=base,
	language=c++,
	morekeywords={std},
	emph={i64,i32,f64,f32}
}

\lstdefinestyle{python}{
	style=base,
	language=Python,
	morekeywords={...},
	emph={int,float,str,bool,list,tuple,dict,set,None}
}

\lstdefinestyle{bash}{
	style=base,
	language=Bash,
	morekeywords={...},
	emph={...}
}

\lstdefinestyle{consola}{
	style=base,
	backgroundcolor=\color{COLOR_CONSOLE},
	keepspaces=true,
	numbers=none,
	xleftmargin=10mm,
	xrightmargin=10mm,
}

\title{Título de la Asignación} % Document title
\author{Nombre} % Document author
\date{\today} % Document date

\begin{document}

\begin{titlepage}
	\centering
	\begin{center}
		\includegraphics[width=3cm]{TMP [T0] - Recursos/logo_ucv.png}
	\end{center}
	{\vspace{-1.5em} \large {UNIVERSIDAD CENTRAL DE VENEZUELA}\par}
	{\vspace{-1em} \large {FACULTAD DE CIENCIAS}\par}
	{\vspace{-1em} \large {ESCUELA DE COMPUTACIÓN}\par}
	{\vspace{-1em} \large {MATERIA}\par}
	\vspace{6.5cm}
	{\LARGE \textbf{TÍTULO DE LA ASIGNACIÓN}\par}
	\vfill
	{\large Nombre Apellido, Nombre Apellido\par}
	{\large \today\par}
\end{titlepage}

\newpage

\section{Introducción}

El presente marco teórico aborda los conceptos fundamentales necesarios para la comprensión de la gestión de procesos, la comunicación entre procesos (IPC) y el manejo de señales en entornos basados en Unix/Linux.

\newpage

\section{Desarrollo}

El siguiente código demuestra una función básica en C++ que calcula el factorial de un número.

\begin{lstlisting}[style=cpp, caption={Ejemplo de Código de C++}]
#include <vector>
#include <algorithm>

int main(int argc, char *argv[]) {
	std::vector<i64> numbers = {5, 3, 8, 6, 2};
	std::sort(numbers.begin(), numbers.end());

	// Apply a lambda function to print the square of each number
	std::for_each(numbers.begin(), numbers.end(), [](i64 x) {
		std::cout << x * x << " ";
	});

	return 0;
}
\end{lstlisting}

El siguiente código demuestra una función básica en C++ que calcula el factorial de un número.

\begin{lstlisting}[style=python, caption={Ejemplo de Código de Python}]
def factorial(n: int) -> int:
	# Base case: factorial of 0 or 1 is 1
	if n == 0 or n == 1:
		return 1
	return n * factorial(n - 1)

if __name__ == "__main__":
	num: int = int(input("Enter a number: "))
	print(f"The factorial of {num} is {factorial(num)}")
\end{lstlisting}

El siguiente código demuestra una función básica en C++ que calcula el factorial de un número.

\begin{lstlisting}[style=bash, caption={Ejemplo de Código de Bash}]
#!/bin/bash

nums=(3 5 7 3 9 5 7)
sum=0

unique=($(printf "%s\n" "${nums[@]}" | sort -n | uniq))
for n in "${unique[@]}"; do ((sum += n)); done

# Display results with parameter expansion
[[ $sum -gt 20 ]] && msg="Greater than 20" || msg="Less than 20"
echo "Unique numbers: ${unique[*]}"
echo "Sum: $sum ($msg)"
\end{lstlisting}

El siguiente código demuestra una función básica en C++ que calcula el factorial de un número.

\begin{lstlisting}[style=consola, caption={Resultado de ejecución del programa \texttt{r1.c}}]
llamando read(3, c, 10).  valor de retorno: 10 bytes leidos
se imprimen los bytes leidos: Jim Plank

invocado read(3, c, 99).  valor de retorno: 12 bytes leidos.
se imprimen los bytes leidos: Claxton 221
\end{lstlisting}

\begin{figure}[h]
	\centering
	\includegraphics[width=1\textwidth]{TMP [T0] - Recursos/img_0001.png}
	\caption{Descripción de la imagen}
\end{figure}

\newpage

\section{Conclusión}

En conclusión, C++ es un lenguaje de programación poderoso y versátil que permite a los desarrolladores crear aplicaciones eficientes y robustas. Los ejemplos presentados en este documento muestran algunas de las características básicas del lenguaje.

\newpage

\section{Referencias Bibliográficas}
\vspace{-4em}
\renewcommand{\refname}{} % Remove the default bibliography title
\renewcommand{\bibname}{} % For book/report classes
\begin{thebibliography}{99}

\bibitem{wiki_ansi}
Wikipedia. (2025, 15 de septiembre). \textit{ANSI C}. \href{https://es.wikipedia.org/wiki/ANSI\_C}{https://es.wikipedia.org/wiki/ANSI\_C}

\end{thebibliography}

\end{document}
